%!TEX program = lualatex
%% -*- fill-column: 120; indent-tabs-mode: nil; tab-width: 2; -*-
\documentclass[
  parskip=half,
  DIV=11,
]{scrartcl}

\usepackage[ngerman]{babel}
\usepackage[utf8]{inputenc}
\usepackage[german=guillemets]{csquotes}
\usepackage{luacode}
\usepackage[dvipsnames, cmyk]{xcolor}
\usepackage{graphicx}
\usepackage{multicol}
\usepackage{tikz}
\usetikzlibrary{positioning, calc}
\usepackage{tikzpagenodes}
\usepackage{fancyvrb}
\usepackage{multicol}

\usepackage{fontspec}
\defaultfontfeatures{Ligatures=TeX}
\setsansfont{SourceSansPro-Light}[
  BoldFont = SourceSansPro,
  ItalicFont = SourceSansPro-Light-it,
]
\renewcommand{\familydefault}{\sfdefault}

\sloppy

\title{
    Das personzentrierte Edukations- und Begleitungsangebot für Menschen mit chronischen Schmerzen
    \textquote{ALGEA}
}
\subject{Exposée}
\author{Sandra Hackenberg}

\begin{document}
\maketitle

\begin{displayquote}\itshape In Deutschland litten aktuell 3,4 Millionen Menschen an schwersten
    chronischen Schmerzen, sagte Kongresspräsident Johannes Horlemann. Dem stünden nur rund 1.200
    ambulant tätige Schmerzmediziner gegenüber. Für eine flächendeckende Versorgung wären mindestens
    10.000 nötig, erläuterte der Präsident der Deutschen Gesellschaft für
    Schmerzmedizin.\footnote{Deutsches Ärzteblatt, 22.07.2020}
\end{displayquote}

\begin{multicols}{2}
Die Versorgung von Patientinnen und Patienten mit chronischen Schmerzen ist vor allem im ambulanten
Bereich noch bei weitem nicht auf dem Niveau, die sie haben sollte. Vor allem fehlt es an Angeboten
mit ganzheitlichem Ansatz, welche die Betroffenen niedrigschwellig und über einen längeren Zeitraum
hinweg begleiten.

Das Angebot setzt sich aus verschiedenen Elementen der Bereiche Erwachsenenpädagogik und Patientenedukation sowie medizinischer und psychologischer Therapie und Prävention zusammen.  Es trägt den
vorläufigen Arbeitstitel ALGEA\footnote{In der griechischen Mythologie sind die Algea die drei
Töchter der Göttin der Zwietracht, Eris, und treten als Personifikation von Kummer, Leid und
Schmerzen auf.} und richtet sich an erwachsene Patientinnen und Patienten mit einer chronischen
Erkrankung, welche als ein Symptom (oft als Hauptsymptom) Schmerzen haben. Darunter fallen zum
Beispiel verschiedene Formen von Kopfschmerzen und Migräne, Rückenschmerzen, Schmerzen durch
Spastiken, bspw. in Folge von Schlaganfällen oder Multipler Sklerose, Nervenschädigungen, bspw.
durch eine Polyneuropathie, nach Infektionen, Unfällen, Operationen und vieles mehr.

Sowohl am Arbeitsplatz als auch im familiären Umfeld erleben Schmerzpatientinnen und -patienten
massive Einschränkungen der Lebensqualität. Diese körperliche und psychische Dauerbelastung führt
oftmals zu einem Teufelskreis aus Schmerzen, Angst, Anspannung und depressiver Verstimmung. Um die Wechselwirkungen zwischen Schmerzerleben, Gedanken, Gefühlen und Verhalten zu verstehen und Veränderungen möglich zu machen ist oft eine intensive
Begleitung und Unterstützung notwendig, insbesondere da die chronische Erkrankung zumeist schon
viele Jahre besteht.

Das Angebot findet in den Praxisräumen der behandelnden Ärztin bzw. des behandelnden Arztes statt,
also in einem Umfeld, das die Betroffenen bereits gut kennen. Es ist über einen Zeitraum von
mindestens 20 Wochen angelegt, um eine langfristige enge Betreuung zu ermöglichen und besteht aus
mehreren Phasen mit Einzel- und Gruppengesprächen im Wechsel, sowie einem Selbstlernmodul. 

Kernmerkmal des Angebots ist neben der langen Dauer der Begleitung vor allem seine Flexibilität. Es
basiert auf den Grundgedanken der personzentrierten Gesprächspsychotherapie nach Carl Rogers. Daher
nimmt es den Menschen in seiner Ganzheit und Individualität wahr und respektiert seine Freiheit und
Eigenverantwortung in hohem Maße. Dementsprechend lässt es das Angebot sowohl inhaltlich als auch
von seiner zeitlichen Gestaltung her ausdrücklich zu, es an die jeweiligen Bedürfnisse der/des
Einzelnen anzupassen. Im Gegensatz zu üblichen klinischen und tagesklinischen Ansätzen legt es den
Betroffenen keinen festen \textquote{Stundenplan} vor. Stattdessen macht es Vorschläge, zeigt Ideen
und Möglichkeiten auf; die Patientin bzw. der Patient erarbeitet mit Unterstützung der professionellen Schmerzbegleiterin,
welche davon am besten zu ihr/ihm und der individuellen Lebenssituation passen.

Langfristiges Ziel des Begleitungsangebots ist es, Betroffene zu befähigen, Anforderungssituationen des
täglichen Lebens, welche durch die Schmerzerkrankung stark beeinträchtigt werden, besser zu
meistern. 

Das Angebot nimmt Schmerzerleben ganzheitlich in den Blick und will den betroffenen Menschen auf
den drei Ebenen Kopf, Herz und Hand\footnote{Die begriffliche Triade geht zurück auf die
ganzheitliche Pädagogik Pestalozzis; derselbe Grundgedanke findet sich aber ebenso in aktuellen
Ansätzen der Schmerzbehandlung nach dem bio-psycho-sozialen Modell bzw. der 3-Säulen-Therapie.}
ansprechen: Kognitiv, sozio-emotional und psychomotorisch. 
\begin{itemize}
\item Kognitive Ebene (\textquote{Kopf}): Gesundheits- und krankheitsbezogenes Wissen, Pathogenese
und Salutogenese, medizinisches Basiswissen zur eigenen Erkrankung, Wirkweise von Medikation,
Zusammenhänge zwischen Körper, Seele und Umwelt
\item Sozio-emotionale Ebene (\textquote{Herz}): Schmerz als seelisches Erleben, Auswirkungen auf
Selbstwertgefühl, Selbstwirksamkeit, soziale Beziehungen; biographische Einflüsse
\item Handlungsebene (\textquote{Hand}): Strategien für das Management der eigenen
Erkrankung, z.B. Therapieplanung, Schmerzbewältigungsstrategien, Entspannungstechniken, körperliche
Aktivierung, Nutzung von Unterstützungsangeboten wie Selbsthilfegruppen, Beratungsstellen etc.
\end{itemize}

Nach meinem bisherigen Kenntnisstand existieren noch keine Behandlungsprogramme, die dieses Mehrebenen-Konzept außerhalb von stationären
Settings über einen längeren Zeitraum für Betroffene anbieten und dabei einen niedrigschwelligen
Zugang (kaum Bürokratie, kurze Wartezeit bis zum Beginn, flexible Gestaltung) in einem bekannten
Umfeld anbieten. Während in klassischen Ansätzen eine typische Aufgabenverteilung an zahlreiche, oft
weitgehend isoliert voneinander arbeitende Fachkräfte, wie Ärztinnen, Psychologinnen,
Physiotherapeuten und Sozialarbeiter stattfindet, bietet ALGEA die Chance, diese Fäden an einem Ort
zusammenlaufen zu lassen und die Patienten ins Zentrum eines Netzwerks aus Unterstützern zu setzen,
ohne ihre Eigenverantwortung zu schmälern. Auf diese Weise kann das Angebot eine Bedarfslücke für
schwer betroffene Schmerzpatientinnen und Schmerzpatienten schließen.

\end{multicols}

\end{document}
